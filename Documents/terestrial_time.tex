\documentclass[]{article}
\usepackage{hyperref}
\usepackage{amsmath}
%\usepackage[backend=bibtex,style=verbose-trad2]{biblatex}

%opening
\title{}
\author{Sven Schmidt}

\begin{document}

\maketitle

\begin{abstract}

\end{abstract}

\section{Introduction}

For many astronomical calculations, the difference between UTC and TT is negligible.
The issue is that UTC is not a monotonically increasing time measure compared to TT
as it is influenced by irregularities in Earth's rotation. These irregularities are
difficult to impossible to predict and can only be measured. NASA publishes tables
with accurate data.

To calculate moon rise and set times, it does make a difference whether we use UTC or
TT. Currently, in early 2022, the difference between UTC and TT is about 68s,~i.e.
TT is ahead of UTC by 68s. This is more than a minute, in which the moon does move
enough to make a difference in rise and set time calculations. According to \cite{BOOK:MontenbruckPfleger}, in
1 minute, the moon moves around 30''.


\section{$UT_{1}$ from $UTC$}

To calculate TT from UTC, we use
\begin{equation}
\label{1}
TT = TAI + 32.184s = UTC + leap(UTC) + 32.184s
\end{equation}
where $leap(UTC)$ are the cumulative leap seconds up to the $UTC$ time.
Document~\href{https://cddis.nasa.gov/archive/products/iers/tai-utc.dat}{tai-utc.dat}
contains leap second data for
\begin{equation}
TAI - UTC = leap(UTC)
\end{equation}
from 1961 to 2017.

We also find
\begin{equation}
\label{2}
TT = UT_{1} + \Delta T
\end{equation}
where $UT_{1}$ is corrected~\href{https://stjarnhimlen.se/comp/time.html#deltat}{UT}.
$\Delta T$ is provided by \href{https://cddis.nasa.gov/archive/products/iers/historic_deltat.data}{NASA}
for historical periods.

In order to calculate $UT_{1}$, we set equ.~\ref{1} and equ.~\ref{2} equal and find
\begin{equation}
UT_{1} + \Delta T = UTC + leap(UTC) + 32.184s
\end{equation}
Solving for $UT_{1}$ yields,
\begin{align}
\label{3}
UT_{1} &= TT - \Delta T \\
       &= UTC + leap(UTC) + 32.184s - \Delta T \\
       &= UTC - \left( \Delta T - leap(UTC) - 32.184s \right)
\end{align}

\section{Calculate $\Delta T$ from $\Delta UT$}

Relation
\begin{equation}
\label{4}
\Delta UT = UT_{1} - UTC
\end{equation}
corrects for variations in UTC.
NASA file \href{https://cddis.nasa.gov/archive/products/iers/finals2000A.all}{finals2000A.all} contains
$\Delta UT$ values for a large range of times and is continuously updated.
We need to calculate $\Delta T$ from $\Delta UT$ to calculate TT from UTC~\ref{3}.
From equ.~\ref{3} and \ref{4} we find
\begin{equation}
\label{5}
\Delta UT = -\Delta T + leap(UTC) + 32.184s
\end{equation}
and finally
\begin{equation}
\label{6}
\Delta T = -\Delta UT + leap(UTC) + 32.184s
\end{equation}

\bibliography{terestrial_time} 
\bibliographystyle{ieeetr}

\end{document}
